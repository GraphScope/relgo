%\enlargethispage{1em}

\section{Conclusions and Discussion}
\label{sec:conclusions}

In this paper, we introduce \name, a converged relational-graph optimization framework designed for SQL/PGQ queries. We formulate the \spjm query skeleton to better analyze and optimize the relational-graph hybrid queries introduced by SQL/PGQ. After discovering that a graph-agnostic approach can result in a larger search space and suboptimal query plans, we design \name to optimize the relational and graph components of \spjm queries using dedicated relational and graph optimization modules, respectively. Additionally, \name incorporates optimization rules, such as \filterrule, which optimize queries across the relational and graph components, further enhancing overall query efficiency. We conduct extensive experiments to evaluate the performance of \name against graph-agnostic baselines, demonstrating its superior performance and confirming the effectiveness of the proposed optimization techniques. 


However, \revise{there is an inevitable gap between relational data and graph data. One important distinction is that not every relation can be clearly identified as either a vertex relation or an edge relation. For example, in the JOB dataset, some tables contain more than two foreign keys, and some of these foreign keys reference the primary key of a vertex relation, making it difficult to definitively classify them as either vertex or edge relations. Another important difference is that joins between relational tables can occur in any order, whereas in plans generated by graph optimizers, vertex relations and edge relations need to alternate. Therefore, the relational optimizer has a larger search space, which can sometimes lead to better results.}

This \revise{problem can be addressed by extending \name to a hybrid optimizer that processes SPJ queries as inputs. Utilizing pre-existing graph indices, the extended \name should determine whether tuples in a relational table can be converted to edges in a graph. 
An intuitive method for converting relational tables to vertices and edges is provided by Boudaoud \cite{Boudaoud2022}.[Can we use citation in this section ?] 
Given an SPJ query, \name transforms arbitrary relational tables into vertices or edges, thereby converting the original SPJ query into an SPJM query. Each matching operator within this query contains a connected pattern graph. The results produced by these matching operators are then joined to yield the final results. Since the costs of matching and relational operators are computable, \name explores various methods for transforming relational tables into vertices and edges to identify the optimal execution plan, and more cost-based global optimization rules can be designed. Meanwhile, extensive global optimizations can lead to increased optimization time overhead. Therefore, it is necessary to carefully balance and select appropriate global and local optimization rules based on the specific circumstances.
We consider extending \name along the aforementioned lines as our future work}.


% Future work includes integrating \name with more widely-used backend databases to expand its applicability and impact.



\balance